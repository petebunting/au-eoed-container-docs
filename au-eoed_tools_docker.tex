\documentclass[authoryear, 11pt, oneside]{report}

\pagestyle{headings} 
\usepackage{geometry} \geometry{a4paper} 
\usepackage[parfill]{parskip} 
\usepackage{natbib} 
\usepackage{graphicx} 
\usepackage{amssymb} 
\usepackage{epstopdf} 
\usepackage{rotating} 
\usepackage{setspace}
\usepackage{lscape}  \DeclareGraphicsRule{.tif}{png}{.png}{`convert #1 `dirname #1`/`basename #1 .tif`.png}
\usepackage{color}
\usepackage[pdftex,bookmarks,plainpages=false]{hyperref}
\usepackage{caption}
\usepackage{subcaption}
\usepackage{framed}

\usepackage{minted}
\usemintedstyle{trac}%perldoc}


%\title{Introduction to the Remote Sensing and GIS Software Library (RSGISLib)}
%\author{Pete Bunting}

\newcommand{\HRule}[1]{\hfill \rule{0.2\linewidth}{#1}} % Horizontal rule at the bottom of the page, adjust width here

\definecolor{grey}{rgb}{0.9,0.9,0.9} % Color of the box surrounding the title - these values can be changed to give the box a different color

\begin{document}



\thispagestyle{empty} % Remove page numbering on this page

%----------------------------------------------------------------------------------------
%	TITLE SECTION
%----------------------------------------------------------------------------------------

\colorbox{grey}{
	\parbox[t]{1.0\linewidth}{
		\centering \fontsize{150pt}{50pt}\selectfont % The first argument for fontsize is the font size of the text and the second is the line spacing - you may need to play with these for your particular title
		\vspace*{0.5cm} % Space between the start of the title and the top of the grey box
		
		\hfill  \\
		\centering Docker:\\
		\centering How to run tools for Remote Sensing and GIS data processing?\\
		\hfill  \\
		
		\vspace*{0.5cm} % Space between the end of the title and the bottom of the grey box
	}
}

\begin{center}
	\includegraphics[width=0.6\columnwidth]{./figures/docker.jpeg}
\end{center}


%----------------------------------------------------------------------------------------

\vfill % Space between the title box and author information

%----------------------------------------------------------------------------------------
%	AUTHOR NAME AND INFORMATION SECTION
%----------------------------------------------------------------------------------------

{\centering \large 
\hfill Pete Bunting \\
\hfill Aberystwyth University \\
\hfill Earth Observation and Ecosystem Dynamics Group \\
\hfill Department of Geography and Earth Sciences \\
\hfill \url{pfb@aber.ac.uk} \\

\HRule{1pt}} % Horizontal line, thickness changed here

%----------------------------------------------------------------------------------------

\begin{center}
	\includegraphics[width=\columnwidth]{./figures/Logos.pdf}
\end{center}

\clearpage % Whitespace to the end of the page




%\maketitle

\begin{framed}
This work (including scripts) is licensed under a Creative Commons Attribution 4.0 International License. To view a copy of this license, visit \url{http://creativecommons.org/licenses/by/4.0/}. 
\end{framed}

\begin{framed}
This document is focus on Docker containers which have been created by the Earth Observation and Ecosystem Dynamics (EOED) research group (\url{https://www.aber.ac.uk/en/dges/research/earth-observation-laboratory/}) at Aberystwyth University (\url{https://www.aber.ac.uk}) to support our research. These containers are focused on the tools and software we use for our data analysis and therefore do not include all possible relevant tools but feel free to use the Dockerfile's as a basis to create your own containers.
\end{framed}

\tableofcontents

\chapter{Docker}

\section{Why Docker?}

Docker (\url{https://www.docker.com}) is a lightweight container system which creates a stand-alone executable package including everything needed to run the package: code, runtime, system tools, system libraries, settings.

A single container can can then be executed on multiple platforms (i.e., Linux, MacOS and Windows) without making any changes to the software or having to re-build the software. The container can also be tagged to create a unique version of the system encapsulated within the container which can be reused at a later point ensuring that a result can be reproduced.

In this case, whether the software is a built via spack (\url{https://spack.readthedocs.io}) or conda-forge (\url{https://conda-forge.org}) we can encapsulate the tools and dependencies within a docker container for re-use. 

\section{Where can Docker not be used?}

Docker cannot be used on a shared system (e.g., HPC) as it requires to run with root access which is not always possible or desirable. However, an alternative is available, which can import Docker containers, called Singularity (\url{https://sylabs.io/singularity/}) and this specifically supports HPC systems. However, Singularity is only available for Linux.

\section{Installing Docker}

\subsection{MacOS and Windows}

To install Docker, login into \url{https://hub.docker.com} and follow the instructions to download and install the software.

\subsection{Linux}

\subsubsection{Ubuntu / Debian}
 Instructions are available from \url{https://docs.docker.com/install/linux/docker-ce/ubuntu/} but if you have not previously had Docker installed then the following commands are used:
 
\begin{minted}[frame=lines]{bash}
sudo apt-get update
sudo apt-get install docker-ce docker-ce-cli containerd.io
\end{minted}

\subsubsection{CentOS}
 Instructions are available from \url{https://docs.docker.com/install/linux/docker-ce/centos/} but if you have not previously had Docker installed then the following commands are used:
 
\begin{minted}[frame=lines]{bash}
sudo yum install docker-ce docker-ce-cli containerd.io
\end{minted}

\subsubsection{Fedora}
 Instructions are available from \url{https://docs.docker.com/install/linux/docker-ce/fedora/} but if you have not previously had Docker installed then the following commands are used:
 
\begin{minted}[frame=lines]{bash}
sudo dnf install docker-ce docker-ce-cli containerd.io
\end{minted}

\section{Useful Images/Containers}

NOTE Terminology: Docker has images and containers. A Docker Image is a set of files which have no state, whereas Docker Container is the instantiation of Docker Image. In other words, Docker Container is the run time instance of images.

On \url{https://hub.docker.com} you will find many useful Docker images which have been built and a ready to use.

The Docker images we will be using for the remainder of this tutorial are from myself and can be browsed at: \url{https://hub.docker.com/u/petebunting}.

\subsection{au-eoed}

This image contains the released version of RSGISLib (\url{https://www.rsgislib.org}) and ARCSI (\url{https://arcsi.remotesensing.info}) and their dependencies, such as GDAL (\url{https://gdal.org}). 

This is the image which most will want to use for satellite imagery analysis.

\subsubsection{URL}
\url{https://hub.docker.com/r/petebunting/au-eoed}

\subsubsection{Installation}

\begin{minted}[frame=lines]{bash}
docker pull petebunting/au-eoed
\end{minted}

\subsection{spdlib}

This image contains the released version of SPDLib (\url{https://spdlib}) its dependencies, such as GDAL (\url{https://gdal.org}). 

This is the image which most will want to use for LiDAR data analysis.

\subsubsection{URL}
\url{https://hub.docker.com/r/petebunting/spdlib}

\subsubsection{Installation}

\begin{minted}[frame=lines]{bash}
docker pull petebunting/spdlib
\end{minted}


\subsection{au-eoed-micmac}

This image contains the released version of MicMac (\url{https://micmac.ensg.eu/index.php/Accueil}) its dependencies and scripts developed by the AU-EOED research group (\url{https://github.com/Ciaran1981/Sfm}) will allow processing of drone photogrammetry data. 

This is the image which most will want to use for processing drone photogrammetry data.

\subsubsection{URL}
\url{https://hub.docker.com/r/petebunting/au-eoed-micmac}

\subsubsection{Installation}

\begin{minted}[frame=lines]{bash}
docker pull petebunting/au-eoed-micmac
\end{minted}


\subsection{au-eoed-dev}

This image contains the development version of RSGISLib (\url{https://www.rsgislib.org}), ARCSI (\url{https://arcsi.remotesensing.info}) EODataDown and other software create by the AU-EOED group and their dependencies, such as GDAL (\url{https://gdal.org}). 

This is the image should not be used in most cases as it changes regularly and at times may contain version of the software which are broken. However, it does contain the very latest versions of the various software and dependencies.

\subsubsection{URL}
\url{https://hub.docker.com/r/petebunting/au-eoed-dev}

\subsubsection{Installation}

\begin{minted}[frame=lines]{bash}
docker pull petebunting/au-eoed-dev
\end{minted}


\chapter{Running Data Analysis}

Once you have pulled your Docker image it is installed on your system, to see which images you have downloaded to your system using the following command: 

\begin{minted}[frame=lines]{bash}
docker image ls
\end{minted}

To see which containers you have running you can use the following command:

\begin{minted}[frame=lines]{bash}
docker ps
\end{minted}

If you find that Docker is using a lot of storage space on your machine then the following command can be used to delete an image from your system:

\begin{minted}[frame=lines]{bash}
docker rmi <IMAGE ID>
\end{minted}

\section{Container Terminal}

The simplest way to use the Docker image us to log into the container on a Terminal prompt. At this point your will have access to the software installed within the Docker image as you would from the Terminal on your own local machine. Running the following command will achieve this (NOTE: type \mintinline{bash}{exit} to leave the container):

\begin{minted}[frame=lines]{bash}
docker run -i -t petebunting/au-eoed /bin/bash
\end{minted}

Once within the container try and run a command such as \mintinline{bash}{gdalinfo --formats} to check the system is working.

However, you will noticed that you do not have access to your files, to get access to your local file system you need to mount this within the Docker container, as show below. NOTE: the variable \mintinline{bash}{${PWD}} is a reference to the current location (i.e., where in your file system you have run the docker image from) this is being mapped on to the \mintinline{bash}{/data} directory within the Docker container.

\begin{minted}[frame=lines]{bash}
docker run -i -t -v ${PWD}:/data petebunting/au-eoed /bin/bash
\end{minted}

From the terminal prompt within the Docker container you can now navigate to the \mintinline{bash}{/data} directory, if you list the contents of the directory you will find the same files as where you execute the \mintinline{bash}{docker run} command.

\begin{minted}[frame=lines]{bash}
cd /data
ls -lh
\end{minted}

You can also specify a specify local path to be mapped, for example:

\begin{minted}[frame=lines]{bash}
docker run -i -t -v /scratch/MyCoolData:/data petebunting/au-eoed /bin/bash
\end{minted}

Please note that you will now have to reference all your paths to \mintinline{bash}{/data} and not the local paths on the machine you are working from. Also, all the data and scripts you want to use also need to be available in \mintinline{bash}{/data}.

\section{ARCSI}

To run ARCSI using the docker image you use the same command as you would have otherwise done but you need to pre-append the Docker command and remember that all the files you are using are relative to the mount point within the Docker container.

\begin{minted}[frame=lines]{bash}
docker run -i -t -v /scratch:/data petebunting/au-eoed arcsi.py -s ls5tm\
-p CLOUDS DOSAOTSGL STDSREF SATURATE TOPOSHADOW FOOTPRINT METADATA \
-o /data/Outputs/ --tmpath /data/tmp --dem /data/UKSRTM_90m.kea \
--k  clouds.kea meta.json sat.kea toposhad.kea valid.kea stdsref.kea \
--stats --format KEA \
-i /data/Input/LT05_L1TP_203024_19950815_20180217_01_T1_MTL.txt
\end{minted}
\newpage
\begin{minted}[frame=lines]{bash}
docker run -i -t -v /scratch:/data petebunting/au-eoed arcsi.py -s sen2 \
-p CLOUDS DOSAOTSGL STDSREF SATURATE TOPOSHADOW FOOTPRINT METADATA SHARP \
-o /data/Outputs  --dem /data/UKSRTM_90m.kea --tmpath /data/tmp \
--k  clouds.kea meta.json sat.kea toposhad.kea valid.kea stdsref.kea \
--stats --format KEA \
-i /data/S2A_MSIL1C_20170617T113321_N0205_R080_T30UVD.SAFE/MTD_MSIL1C.xml
\end{minted}

\section{Using GDAL Tools}

Using one of the GDAL tools is similar to the ARCSI, in that the commands are all the same but you need to update file paths to be relative to the mount point in the Docker container. For example:

\begin{minted}[frame=lines]{bash}
docker run -i -t -v /scratch:/data petebunting/au-eoed gdal_translate \
-of GTIFF /data/input_img.kea /data/output_img.tif
\end{minted}

\section{Running Python (RSGISLib)}

Again, the change which is needed related to the file paths either being inputting into the python script. For example, the following python script (saved as \mintinline{bash}{calc_ndvi.py})

\begin{minted}[frame=lines, linenos]{python}
import rsgislib.imagecalc

img = '/data/landsat_img.kea'
out_img = '/data/landsat_ndvi.kea'
rsgislib.imagecalc.calcindices.calcNDVI(img, 3, 4, out_img)

\end{minted}

Can be executed using the following Docker command:

\begin{minted}[frame=lines]{bash}
docker run -i -t -v /scratch:/data petebunting/au-eoed \
python /data/calc_ndvi.py
\end{minted}


\end{document}  
